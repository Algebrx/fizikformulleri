\documentclass[a4paper, 11pt, titlepage]{article}
\usepackage[a4paper, left=20mm, right=20mm, top=20mm, bottom=20mm]{geometry}
\title{AYT Fizik Formülleri}
\author{İsa Cebir}
\date{24.05.2025}
\usepackage[utf8]{inputenc}
\usepackage[turkish]{babel}
\usepackage{lipsum}
\usepackage[hidelinks]{hyperref}
\usepackage{fancyhdr}
\usepackage[none]{hyphenat}
\usepackage[x11names]{xcolor}
\usepackage{sectsty}
\definecolor{blue}{HTML}{0099ff}
\definecolor{grey}{HTML}{737373}
\sectionfont{\color{blue}}
\begin{document}
\pagestyle{fancy}
\maketitle
\thispagestyle{empty}
\fancyhf{} 
\fancyhead[L]{AYT Fizik Formülleri}
\fancyfoot[R]{\thepage}
\fancyhead[R]{İsa Cebir}
\renewcommand{\headrulewidth}{0.2pt}
\renewcommand{\footrulewidth}{0.2pt}

\twocolumn

\tableofcontents
\newpage
\twocolumn

\section{Hız-Zaman Denklemi}
\[
\vec{v} = \vec{v}_0 + \vec{a} \cdot t
\]
\begin{itemize}
  \item $\vec{v}$ : Son hız
  \item $\vec{v}_0$ : Başlangıç hızı
  \item $\vec{a}$ : İvme
  \item $t$ : Zaman
\end{itemize}

\section{Konum-Zaman Denklemi}
\[
\vec{x} = \vec{x}_0 + \vec{v}_0 \cdot t + \frac{1}{2} \vec{a} \cdot t^2
\]
\begin{itemize}
  \item $\vec{x}$ : Son konum
  \item $\vec{x}_0$ : Başlangıç konumu
  \item $\vec{v}_0$ : Başlangıç hızı
  \item $\vec{a}$ : İvme
  \item $t$ : Zaman
\end{itemize}

\section{Zamansız Hız Denklemi}
\[
v^2 = v_0^2 + 2a \cdot \Delta x
\]
\begin{itemize}
  \item $v$ : Son hızın büyüklüğü
  \item $v_0$ : Başlangıç hızının büyüklüğü
  \item $a$ : İvme
  \item $\Delta x$ : Yer değiştirme
\end{itemize}

\section{İş}
\[
W = \vec{F} \cdot \vec{d} = F d \cos \theta
\]
\begin{itemize}
  \item $W$ : İş
  \item $\vec{F}$ : Kuvvet
  \item $\vec{d}$ : Yer değiştirme
  \item $\theta$ : Kuvvet ile yer değiştirme arasındaki açı
\end{itemize}

\section{Güç}
\[
P = \frac{W}{t}
\]
\begin{itemize}
  \item $P$ : Güç
  \item $W$ : Yapılan iş
  \item $t$ : Zaman
\end{itemize}

\section{Kinetik Enerji}
\[
K = \frac{1}{2} m v^2
\]
\begin{itemize}
  \item $K$ : Kinetik enerji
  \item $m$ : Kütle
  \item $v$ : Hız
\end{itemize}

\section{Potansiyel Enerji}
\[
U = m g h
\]
\begin{itemize}
  \item $U$ : Potansiyel enerji
  \item $m$ : Kütle
  \item $g$ : Yerçekimi ivmesi
  \item $h$ : Yükseklik
\end{itemize}

\section{Yay Potansiyeli}
\[
U = \frac{1}{2} k x^2
\]
\begin{itemize}
  \item $U$ : Yayın potansiyel enerjisi
  \item $k$ : Yay sabiti
  \item $x$ : Yayın uzama veya sıkışma miktarı
\end{itemize}

\section{Kuvvet}
\[
\vec{F} =m\cdot\vec{a}
\]
\begin{itemize}
\item $\vec{F}$ : Kuvvet
\item $m$ : Kütle
\item $\vec{a}$ : İvme
\end{itemize}

\section{Tork}
\[
\vec{\tau} = \vec{d} \times \vec{F}
\]
\begin{itemize}
  \item $\vec{\tau}$ : Tork
  \item $\vec{d}$ : Denge merkezine olan dik uzaklık
  \item $\vec{F}$ : Kuvvet
\end{itemize}

\section{İtme}
\[
\vec{I} = \vec{F}_{\mathrm{net}} \times \Delta t
\]
\begin{itemize}
  \item $\vec{I}$ : İtme
  \item $\vec{F}_{\mathrm{net}}$ : Uygulanan kuvvet
  \item $\Delta t$: Geçen zaman
\end{itemize}


\section{Çizgisel Momentum}
\[
\vec{p} = m \cdot \vec{v}
\]
\begin{itemize}
  \item $\vec{p}$ : Çizgisel momentum
  \item $m$ : Kütle
  \item $\vec{v}$ : Hız
\end{itemize}

\section{Elektriksel Kuvvet}
\[
\vec{F} = k \frac{q_1 \cdot q_2}{r^2}
\]
\begin{itemize}
  \item $\vec{F}$ : Elektriksel kuvvet
  \item $q_1$, $q_2$ : Yükler
  \item $r$ : Yükler arası uzaklık
  \item $k$ : Coulomb sabiti
\end{itemize}

\section{Elektriksel Alan}
\[
\vec{E} = \frac{\vec{F}}{q}
\]
\begin{itemize}
  \item $\vec{E}$ : Elektriksel alan
  \item $\vec{F}$ : Elektriksel kuvvet
  \item $q$ : Test yükü
\end{itemize}

\section{Elektriksel Potansiyel Enerji}
\[
U = \pm k \frac{q_1 \cdot q_2}{r}
\]
\begin{itemize}
  \item $U$ : Elektriksel potansiyel enerji
  \item $q_1$, $q_2$ : Yükler
  \item $r$ : Yükler arası uzaklık
  \item $k$ : Coulomb sabiti
\end{itemize}

\section{Elektriksel Potansiyel}
\[
V = \pm k \frac{Q}{r}
\]
\begin{itemize}
  \item $V$ : Elektriksel potansiyel
  \item $Q$ : Noktasal yük
  \item $r$ : Uzaklık
  \item $k$ : Coulomb sabiti
\end{itemize}

\section[P.L.A. Elektriksel Alan]{Paralel Levhalar Arasında \\ Elektriksel Alan}
\[
E = \frac{V}{d}
\]
\begin{itemize}
  \item $E$ : Elektriksel alan
  \item $V$ : Potansiyel farkı
  \item $d$ : Levhalar arası mesafe
\end{itemize}

\section[P.L.A. Parçacığa Etki Eden Kuvvet]{Paralel Levhalar Arasında \\Parçacığa Etki Eden Kuvvet}
\[
F = q\cdot E
\]
\begin{itemize}
  \item $F$ : Elektriksel kuvvet
  \item $q$ : Yüklü parçacık
  \item $E$ : Elektriksel alan
\end{itemize}

\section[P.L.A. Potansiyel Farkı]{Paralel Levhalar Arasında \\Potansiyel Farkı}
\[
V = \pm E \cdot d
\]
\begin{itemize}
  \item $V$ : Potansiyel farkı
  \item $E$ : Elektriksel alan
  \item $d$ : Levhalar arası mesafe
\end{itemize}

\section[P.L.A. Kinetik Enerji Değişimi]{Paralel Levhalar Arasında \\Kinetik Enerji Değişimi}
\[
\Delta K = \pm q \Delta V
\]
\begin{itemize}
  \item $\Delta K$ : Kinetik enerji değişimi
  \item $q$ : Yük
  \item $\Delta V$ : Potansiyel farkı
\end{itemize}

\section{Sığaç Kapasite Formülü}
\[
C = \frac{\varepsilon A}{d}
\]
\begin{itemize}
  \item $C$ : Kapasitans (sığa)
  \item $\varepsilon$ : Ortamın elektriksel geçirgenliği
  \item $A$ : Plaka yüzey alanı
  \item $d$ : Plaklar arası mesafe
\end{itemize}

\section{Fotonun Enerjisi}
\[
E = h \cdot f = \frac{h c}{\lambda}
\]
\begin{itemize}
  \item $E$ : Fotonun enerjisi
  \item $h$ : Planck sabiti
  \item $f$ : Frekans
  \item $\lambda$ : Dalga boyu
  \item $c$ : Işık hızı
\end{itemize}

\section{Fotoelektrik Denklemi}
\[
E_g = E_0 + E_{e^-}
\]
\begin{itemize}
  \item $E_g$ : Gelen fotonun enerjisi
  \item $E_0$ : Metalin eşik enerjisi
  \item $E_{e^-}$ : Saçılan elektronun kinetik enerjisi
\end{itemize}

\section{De Broglie Denklemi}
\[
\lambda = \frac{h}{P}
\]
\begin{itemize}
  \item $\lambda$ : Parçacığın dalga boyu
  \item $h$ : Planck sabiti
  \item $P$ : Momentun
\end{itemize}
\end{document}
