\documentclass[a4paper, 11pt, titlepage]{article}
\usepackage{accents}
\usepackage[a4paper, left=20mm, right=20mm, top=20mm, bottom=20mm]{geometry}
\title{AYT Fizik Formülleri}
\author{İsa Cebir}
\date{24.05.2025}
\usepackage[T1]{fontenc}
\usepackage[utf8]{inputenc}
\usepackage[turkish]{babel}
\usepackage{lipsum}
\usepackage[hidelinks]{hyperref}
\usepackage{fancyhdr}
\usepackage[none]{hyphenat}
\usepackage[x11names]{xcolor}
\usepackage{sectsty}
\definecolor{blue}{HTML}{0099ff}
\definecolor{grey}{HTML}{787878}
\sectionfont{\color{black}}
\subsectionfont{\color{black}}
\color{grey}
\begin{document}
\pagestyle{fancy}
\maketitle
\thispagestyle{empty}
\fancyhf{} 
\fancyhead[L]{AYT Fizik Formülleri}
\fancyfoot[R]{\thepage}
\fancyhead[R]{İsa Cebir}
\renewcommand{\headrulewidth}{0.2pt}
\renewcommand{\footrulewidth}{0.2pt}

\twocolumn

\tableofcontents
\newpage
\twocolumn

\section{Hareket}

\subsection{Hız-Zaman Denklemi}
\[
\vec{v} = \vec{v}_0 + \vec{a} \cdot t
\]
\begin{itemize}
  \item $\vec{v}$ : Son hız
  \item $\vec{v}_0$ : Başlangıç hızı
  \item $\vec{a}$ : İvme
  \item $t$ : Zaman
\end{itemize}

\subsection{Konum-Zaman Denklemi}
\[
\vec{x} = \vec{x}_0 + \vec{v}_0 \cdot t + \frac{1}{2} \vec{a} \cdot t^2
\]
\begin{itemize}
  \item $\vec{x}$ : Son konum
  \item $\vec{x}_0$ : Başlangıç konumu
  \item $\vec{v}_0$ : Başlangıç hızı
  \item $\vec{a}$ : İvme
  \item $t$ : Zaman
\end{itemize}


\subsection{Zamansız Hız Denklemi}
\[
v^2 = v_0^2 + 2a \cdot \Delta x
\]
\begin{itemize}
  \item $v$ : Son hızın büyüklüğü
  \item $v_0$ : Başlangıç hızının büyüklüğü
  \item $a$ : İvme
  \item $\Delta x$ : Yer değiştirme
\end{itemize}

\subsection[Serbest Düşmede Yer Değiştirme]{Serbest Düşmede Yer \\Değiştirme}
\[
h = \frac{1}{2} g t^2
\]
\begin{itemize}
  \item $h$ : Yükseklik
  \item $g$ : Yerçekimi ivmesi
  \item $t$ : Zaman
\end{itemize}

\section{Enerji}
\subsection{İş}
\[
W = \vec{F} \cdot \vec{d} = F d \cos \theta
\]
\begin{itemize}
  \item $W$ : İş
  \item $\vec{F}$ : Kuvvet
  \item $\vec{d}$ : Yer değiştirme
  \item $\theta$ : Kuvvet ile yer değiştirme arasındaki açı
\end{itemize}

\subsection{Güç}
\[
P = \frac{W}{t}
\]
\begin{itemize}
  \item $P$ : Güç
  \item $W$ : Yapılan iş
  \item $t$ : Zaman
\end{itemize}

\subsection{Kinetik Enerji}
\[
K = \frac{1}{2} m v^2
\]
\begin{itemize}
  \item $K$ : Kinetik enerji
  \item $m$ : Kütle
  \item $v$ : Hız
\end{itemize}

\subsection{Potansiyel Enerji}
\[
U = m g h
\]
\begin{itemize}
  \item $U$ : Potansiyel enerji
  \item $m$ : Kütle
  \item $g$ : Yerçekimi ivmesi
  \item $h$ : Yükseklik
\end{itemize}

\subsection{Yay Potansiyeli}
\[
U = \frac{1}{2} k x^2
\]
\begin{itemize}
  \item $U$ : Yayın potansiyel enerjisi
  \item $k$ : Yay sabiti
  \item $x$ : Yayın uzama veya sıkışma miktarı
\end{itemize}

\section{Tork ve Denge}



\subsection{Tork}
\[
\vec{\tau} = \vec{d} \times \vec{F}
\]
\begin{itemize}
  \item $\vec{\tau}$ : Tork
  \item $\vec{d}$ : Denge merkezine olan dik uzaklık
  \item $\vec{F}$ : Kuvvet
\end{itemize}

\subsection{Kesişen Kuvvet Dengesi}
\[
\frac{F_1}{\sin \alpha} = \frac{F_2}{\sin \beta} = \frac{F_3}{\sin \gamma}
\]
\begin{itemize}
  \item $F_1, F_2, F_3$ : Kuvvet büyüklükleri
  \item $\alpha, \beta, \gamma$ : Kuvvetlerin karşılıklı açıları
\end{itemize}


\subsection[Kütle Merkezinin Koordinatlar]{Kütle Merkezinin \\Koordinatları}
\[
x_{\mathrm{cm}} = \frac{m_1 x_1 + m_2 x_2 + \cdots + m_n x_n}{m_1 + m_2 + \cdots + m_n}
\]
\[
y_{\mathrm{cm}} = \frac{m_1 y_1 + m_2 y_2 + \cdots + m_n y_n}{m_1 + m_2 + \cdots + m_n}
\]
\begin{itemize}
  \item $m_i$ : $i$-inci kütle
  \item $x_i, y_i$ : $i$-inci kütlenin koordinatları
  \item $x_{\mathrm{cm}}, y_{\mathrm{cm}}$ : Kütle merkezinin koordinatları
\end{itemize}
\section{İtme ve Momentum}
\subsection{İtme}
\[
\vec{I} = \vec{F}_{\mathrm{net}} \times \Delta t
\]
\begin{itemize}
  \item $\vec{I}$ : İtme
  \item $\vec{F}_{\mathrm{net}}$ : Uygulanan kuvvet
  \item $\Delta t$: Geçen zaman
\end{itemize}



\subsection{Çizgisel Momentum}
\[
\vec{P} = m \cdot \vec{v}
\]
\begin{itemize}
  \item $\vec{P}$ : Çizgisel momentum
  \item $m$ : Kütle
  \item $\vec{v}$ : Hız
\end{itemize}
\section{Elektrik}

\subsection{Elektriksel Alan}
\[
\vec{E} = \frac{\vec{F}}{q}
\]
\begin{itemize}
  \item $\vec{E}$ : Elektriksel alan
  \item $\vec{F}$ : Elektriksel kuvvet
  \item $q$ : Test yükü
\end{itemize}


\subsection{Elektriksel Kuvvet}
\[
\vec{F} = k \frac{q_1 \cdot q_2}{r^2}
\]
\begin{itemize}
  \item $\vec{F}$ : Elektriksel kuvvet
  \item $q_1$, $q_2$ : Yükler
  \item $r$ : Yükler arası uzaklık
  \item $k$ : Coulomb sabiti
\end{itemize}

\subsection{Elektriksel Potansiyel Enerji}
\[
U = \pm k \frac{q_1 \cdot q_2}{r}
\]
\begin{itemize}
  \item $U$ : Elektriksel potansiyel enerji
  \item $q_1$, $q_2$ : Yükler
  \item $r$ : Yükler arası uzaklık
  \item $k$ : Coulomb sabiti
\end{itemize}

\subsection{Elektriksel Potansiyel}
\[
V = \pm k \frac{q}{r}
\]
\begin{itemize}
  \item $V$ : Elektriksel potansiyel
  \item $q$ : Noktasal yük
  \item $r$ : Uzaklık
  \item $k$ : Coulomb sabiti
\end{itemize}

\subsection[P.L.A. Elektriksel Alan]{Paralel Levhalar Arasında \\ Elektriksel Alan}
\[
E = \frac{V}{d}
\]
\begin{itemize}
  \item $E$ : Elektriksel alan
  \item $V$ : Potansiyel farkı
  \item $d$ : Levhalar arası mesafe
\end{itemize}

\subsection[P.L.A. Parçacığa Etki Eden Kuvvet]{Paralel Levhalar Arasında \\Parçacığa Etki Eden Kuvvet}
\[
F = q\cdot \frac{V}{d}
\]
\begin{itemize}
  \item $F$ : Elektriksel kuvvet
  \item $q$ : Yüklü parçacık
  \item $V$ : Potansiyel fark
  \item $d$ : Levhalar arası mesafe
\end{itemize}

\subsection[P.L.A. Kinetik Enerji Değişimi]{Paralel Levhalar Arasında \\Kinetik Enerji Değişimi}
\[
\Delta K = \pm q \Delta V
\]
\begin{itemize}
  \item $\Delta K$ : Kinetik enerji değişimi
  \item $q$ : Yük
  \item $\Delta V$ : Potansiyel farkı
\end{itemize}

\subsection[Yük-Gerilim Bağıntısı]{Kondansatörlerde Yük-Gerilim \\Bağıntısı}
\[
q = C \cdot V
\]
\begin{itemize}
  \item $q$ : Yük
  \item $C$ : Kapasitans (sığa)
  \item $V$ : Gerilim (potansiyel farkı)
\end{itemize}

\subsection{Sığaç Kapasite Formülü}
\[
C = \varepsilon \frac{A}{d}
\]
\begin{itemize}
  \item $C$ : Kapasitans (sığa)
  \item $\varepsilon$ : Ortamın elektriksel geçirgenliği
  \item $A$ : Plaka yüzey alanı
  \item $d$ : Plaklar arası mesafe
\end{itemize}


\section{Manyetizma}

\subsection[Akım Taşıyan Düz Telde Manyetik Alan]{Akım Taşıyan Düz Telin \\Oluşturduğu Manyetik Alan}
\[
B = 2K\frac{i}{d}
\]
\begin{itemize}
  \item $B$ : Manyetik alan şiddeti
  \item $K$ : Manyetik alan sabiti
  \item $i$ : Telden geçen akım
  \item $r$ : Tele olan uzaklık
\end{itemize}

\subsection{Bobin İçindeki Manyetik Alan}
\[
B = K \cdot \frac{4\pi Ni}{\ell}
\]
\begin{itemize}
  \item $B$ : Manyetik alan şiddeti
  \item $K$ : Manyetik alan sabiti
  \item $i$ : Telden geçen akım
  \item $\ell$ : Sarım uzunluğu
  \item $N$ : Sarım sayısı
\end{itemize}
\subsection[Tele Etki Eden Manyetik Kuvvet]{Akım Taşıyan Tele Manyetik \\Alanda Etki Eden Kuvvet}
\[
F = B \cdot i \cdot L \cdot \sin \theta
\]
\begin{itemize}
  \item $F$ : Manyetik kuvvet
  \item $i$ : Akım
  \item $L$ : Telin uzunluğu
  \item $B$ : Manyetik alan
  \item $\theta$ : Akım yönü ile manyetik alan arasındaki açı
\end{itemize}

\subsection[Hareket Eden Yüklü Parçacığa Etki Eden Manyetik Kuvvet]{Manyetik Alan İçinde Hareket \\Eden Yüklü Parçacığa Etki Eden Kuvvet}
\[
F = q \cdot v \cdot B \cdot \sin \theta
\]
\begin{itemize}
  \item $F$ : Manyetik kuvvet
  \item $q$ : Yük
  \item $v$ : Parçacığın hızı
  \item $B$ : Manyetik alan
  \item $\theta$ : Hız vektörü ile manyetik alan arasındaki açı
\end{itemize}

\subsection[Çekilen Telde İndüksiyon EMK]{Manyetik Alanda Çekilen Telin \\Uçları Arasında Oluşan İndüksiyon \\Elektromotor Kuvveti}
\[
\mathcal{E} = B \cdot L \cdot v
\]
\begin{itemize}
  \item $\mathcal{E}$ : İndüksiyon elektromotor kuvveti (emk)
  \item $B$ : Manyetik alan
  \item $L$ : Tel uzunluğu
  \item $v$ : Telin çekilme hızı
\end{itemize}

\subsection[Döndürülen Telde İndüksiyon EMK]{Manyetik Alanda Döndürülen Bir \\Telin Uçları Arasında Oluşan \\İndüksiyon Elektromotor Kuvveti}
\[
\mathcal{E} = \frac{B\omega\ell^2}{2} \qquad \mathcal{E} = \frac{Bv\ell}{2}
\]
\begin{itemize}
  \item $\mathcal{E}$ : İndüksiyon emk
  \item $N$ : Sarım sayısı
  \item $B$ : Manyetik alan
  \item $A$ : Alan
  \item $\omega$ , $v$ : Açısal veya çizgisel hız
  \item $t$ : Zaman
\end{itemize}

\subsection{Manyetik Akı}
\[
\Phi = B \cdot A \cdot \cos \theta
\]
\begin{itemize}
  \item $\Phi$ : Manyetik akı
  \item $B$ : Manyetik alan
  \item $A$ : Alan
  \item $\theta$ : Yüzey normali ile manyetik alan arasındaki açı
\end{itemize}

\subsection{Alternatif Akımda Etkin Değer}
\[
I_{\mathrm{etkin}} = \frac{I_{\mathrm{maks}}}{\sqrt{2}}, \quad V_{\mathrm{etkin}} = \frac{V_{\mathrm{maks}}}{\sqrt{2}}
\]
\begin{itemize}
  \item $I_{\mathrm{etkin}}$ : Etkin akım
  \item $I_{\mathrm{maks}}$ : Maksimum akım
  \item $V_{\mathrm{etkin}}$ : Etkin gerilim
  \item $V_{\mathrm{maks}}$ : Maksimum gerilim
\end{itemize}

\subsection{Bobinin İndüktif Reaktansı}
\[
X_L = 2 \pi f L
\]
\begin{itemize}
  \item $X_L$ : Bobinin indüktif reaktansı
  \item $f$ : Frekans
  \item $L$ : Bobinin endüktansı
\end{itemize}

\subsection{Sığacın Kapasitif Reaktansı}
\[
X_C = \frac{1}{2 \pi f C}
\]
\begin{itemize}
  \item $X_C$ : Sığacın kapasitif reaktansı
  \item $f$ : Frekans
  \item $C$ : Kapasitans
\end{itemize}

\subsection{Transformatörler}
\[
\frac{N_1}{N_2} = \frac{V_1}{V_2} = \frac{i_2}{i_1}
\]
\begin{itemize}
  \item $N_1$, $N_2$ : Primer ve sekonder sarım sayıları
  \item $V_1$, $V_2$ : Primer ve sekonder gerilimler
  \item $i_1$, $i_2$ : Primer ve sekonder akımlar
\end{itemize}
\section{Çembersel Hareket}
\subsection{Açısal Hız}
\[
\omega = 2 \pi f
\]
\begin{itemize}
  \item $\omega$ : Açısal hız 
  \item $f$ : Frekans
\end{itemize}

\subsection{Çizgisel Hız}
\[
v = 2 \pi f r
\]
\begin{itemize}
  \item $v$ : Çizgisel hız
  \item $r$ : Yarıçap
  \item $f$ : Frekans
\end{itemize}

\subsection{Merkezcil İvme}
\[
a = \frac{v^2}{r} \qquad a = \omega^2 \cdot r
\]
\begin{itemize}
  \item $a$ : Merkezcil ivme
  \item $v$ : Çizgisel hız
  \item $\omega$ : Açılsal hız
  \item $r$ : Yarıçap
\end{itemize}

\subsection{Merkezcil Kuvvet}
\[
F = \frac{m v^2}{r}
\]
\begin{itemize}
  \item $F$ : Merkezcil kuvvet
  \item $m$ : Kütle
  \item $v$ : Çizgisel hız
  \item $r$ : Yarıçap
\end{itemize}

\subsection[Minimum Sürtünme Katsayısı]{Minimum Sürtünme \\Katsayısı (Savrulmama)}
\[
k_{\mathrm{min}} = \frac{\omega^2 r}{g}
\]
\begin{itemize}
  \item $\mu_{\mathrm{min}}$ : Minimum sürtünme katsayısı
  \item $v$ : Açısal hız
  \item $r$ : Dönme yarıçapı
  \item $g$ : Yerçekimi ivmesi
\end{itemize}

\subsection[Minimum  Açılsal Hız]{Minimum Açısal Hız \\(Savrulmama)}
\[
\omega_{\mathrm{min}} = \sqrt{\frac{k g}{r}}
\]
\begin{itemize}
  \item $\omega_{\mathrm{min}}$ : Minimum açısal hız
  \item $k$ : Sürtünme katsayısı
  \item $g$ : Yerçekimi ivmesi
  \item $r$ : Yarıçap
\end{itemize}

\subsection[Eğimli Virajda Maksimum Hız]{Eğimli Virajda Maksimum \\ Hız}
\[
v_{\mathrm{max}} = \sqrt{g r \tan \alpha}
\]
\begin{itemize}
  \item $v_{\mathrm{max}}$ : Maksimum hız
  \item $g$ : Yerçekimi ivmesi
  \item $r$ : Yarıçap
  \item $\alpha$ : Eğim açısı
\end{itemize}

\subsection{Açısal Momentum}
\[
L = m \cdot v \cdot r
\]
\begin{itemize}
  \item $L$ : Açısal momentum
  \item $m$ : Kütle
  \item $v$ : Hız
  \item $r$ : Yarıçap
\end{itemize}


\subsection{Eylemsizlik Momenti}
\[
I = m \cdot r^2
\]
\begin{itemize}
  \item $I$ : Eylemsizlik momenti
  \item $m$ : Kütle
  \item $r$ : Yarıçap
\end{itemize}
\section[Kütle Çekimi ve Kepler Kanunları]{Kütle Çekimi ve Kepler \\Kanunları}
\subsection{Kütle Çekim Kuvveti}
\[
F = G \frac{m_1 \cdot m_2}{r^2}
\]
\begin{itemize}
  \item $F$ : Kütleler arasındaki çekim kuvveti
  \item $G$ : Evrensel çekim sabiti
  \item $m_1, m_2$ : Kütleler
  \item $r$ : Kütleler arası uzaklık
\end{itemize}

\subsection{Periyotlar Kanunu}
\[
\frac{r^3}{T^2} = \mathrm{Sabit}
\]
\begin{itemize}
  \item $r$ : Yörüngenin yarıçapı
  \item $T$ : Dolanım periyodu
\end{itemize}


\subsection{Yerçekimi İvmesi}
\[
g = G \frac{m}{r^2}
\]
\begin{itemize}
  \item $g$ : Yerçekimi ivmesi
  \item $G$ : Evrensel çekim sabiti
  \item $m$ : Dünya'nın (ya da gökcisminin) kütlesi
  \item $r$ : Merkezden olan uzaklık
\end{itemize}
 \section{Basit Harmonik Hareket}
 \subsection[Hızın Uzanıma Bağlı Değişimi]{Hızın Uzanıma Bağlı \\Değişimi}
 \[
 v = \omega \sqrt{R^2 - x^2}
 \]
 \begin{itemize}
   \item $v$ : Anlık hız
   \item $\omega$ : Açısal frekans
   \item $R$ : Maksimum genlik (uzanım)
   \item $x$ : Anlık uzanım
 \end{itemize}

\subsection{İvmenin Uzanıma Bağlı Değişimi}
\[
a = -\omega^2 x
\]
\begin{itemize}
  \item $a$ : Anlık ivme
  \item $\omega$ : Açısal frekans
  \item $x$ : Anlık uzanım
\end{itemize}

\subsection{Maksimum Hız}
\[
v_{\mathrm{max}} = \omega R
\]
\begin{itemize}
  \item $v_{\mathrm{max}}$ : Maksimum hız
  \item $\omega$ : Açısal frekans
  \item $R$ : Genlik
\end{itemize}

\subsection{Maksimum İvme}
\[
a_{\mathrm{max}} = \omega^2 R
\]
\begin{itemize}
  \item $a_{\mathrm{max}}$ : Maksimum ivme
  \item $\omega$ : Açısal frekans
  \item $R$ : Genlik
\end{itemize}

\subsection{Basit Sarkaçta Periyot}
\[
T = 2\pi \sqrt{\frac{L}{g}}
\]
\begin{itemize}
  \item $T$ : Periyot
  \item $L$ : Sarkaç uzunluğu
  \item $g$ : Yerçekimi ivmesi
\end{itemize}


\subsection{Yaylı Sarkaçta Periyot}
\[
T = 2\pi \sqrt{\frac{m}{k}}
\]
\begin{itemize}
  \item $T$ : Periyot
  \item $m$ : Kütle
  \item $k$ : Yay sabiti
\end{itemize}
\section{Modern Fizik}
\subsection{Fotonun Enerjisi}
\[
E = h \cdot f = \frac{h c}{\lambda}
\]
\begin{itemize}
  \item $E$ : Fotonun enerjisi
  \item $h$ : Planck sabiti
  \item $f$ : Frekans
  \item $\lambda$ : Dalga boyu
  \item $c$ : Işık hızı
\end{itemize}

\subsection{Fotoelektrik Denklemi}
\[
E_g = E_0 + E_{e^-}
\]
\begin{itemize}
  \item $E_g$ : Gelen fotonun enerjisi
  \item $E_0$ : Metalin eşik enerjisi
  \item $E_{e^-}$ : Saçılan elektronun kinetik enerjisi
\end{itemize}

\subsection{De Broglie Denklemi}
\[
\lambda = \frac{h}{P}
\]
\begin{itemize}
  \item $\lambda$ : Parçacığın dalga boyu
  \item $h$ : Planck sabiti
  \item $P$ : Momentum
\end{itemize}
\end{document}





%5.6